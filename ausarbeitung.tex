
% ---------------------------------------------
% Autor:        UOS, Arbeitsgruppe Software Engineering und CAU Kiel
%               mit Verbesserungen von Valentin Bruder (26.06.13)
% Beschreibung: Vorlage für Seminarausarbeitung
% Dateiname:    ausarbeitung.tex 
% Projekt:      Seminare
% erstellt am:  24.04.2013
% geändert am:  $Date$
% ---------------------------------------------

%%%%%%%%%%%%%%%%%%%%%%%%%%%%%%%%%%%%%%%%%%%%
%%%             Achtung!!                %%%
%%% In diesem Beispiel werden Pakete ver-%%%
%%% wendet, die ggf. nachinstalliert wer-%%%
%%% den müssen!                          %%%
%%%%%%%%%%%%%%%%%%%%%%%%%%%%%%%%%%%%%%%%%%%%


%%%%%%%%%%%%%%%%%%%%%%%%%%%%%%%%%%%%%%%%%%%%
%%% Es wird KOMA-Script verwendet        %%%
%%% welches in den meisten TeX-          %%%
%%% Distributionen enthalten ist!        %%%
%%%%%%%%%%%%%%%%%%%%%%%%%%%%%%%%%%%%%%%%%%%%
%\documentclass[a4paper, 11pt, DIV=11, listof=numbered, bibliography=numbered, pointlessnumbers]{scrartcl}
\documentclass[a4paper, 11pt, DIV=11, listof=numbered, numbers=noenddot]{scrartcl}

%%%%%%%%%%%%%%%%%%%%%%%%%%%%%%%%%%%%%%%%%%%%%%
%%%     benötigte Pakete                   %%%
%%%%%%%%%%%%%%%%%%%%%%%%%%%%%%%%%%%%%%%%%%%%%%
\usepackage[utf8]{inputenc}     % Paket für Eingabekodierung (utf8 für die meisten aktuellen Linux-Distributionen): Umlaute
%\usepackage[latin1]{inputenc}  % Eingabekodierung für Windows falls UTF8 nicht unterstützt wird
\usepackage{ngerman}		% Deutsche Spracheigenheiten
\usepackage{amsmath}		% AMS Mathe und Symbolpakete für mehr Möglichkeiten
\usepackage{amsfonts}
%%\usepackage{amssymb}
% \usepackage{floatflt}
% \usepackage{float}
\usepackage{graphicx}		% Grafiken einbinden
\usepackage{xcolor}
\usepackage{hyperref}		% Erzeugt Verlinkungen im Dokument
\usepackage{url}			% erlaubt das einfache Schreiben von URLs
\usepackage{listings}		% für Listings (ftp://ftp.ctan.org/tex-archive/macros/latex/contrib/listings/listings.pdf)
\usepackage[T1]{fontenc} 	% Wichtig für Umlaute in PDFs
\usepackage{lmodern}		% Für bessere Schriftdarstellung trotz fontenc


%%%%%%%%%%%%%%%%%%%%%%%%%%%%%%%%%%%%%%%%%%%%%%
%%%      Weitere sinnvolle Pakete          %%%
%%%      ggf. Kommentar entfernen          %%%
%%%%%%%%%%%%%%%%%%%%%%%%%%%%%%%%%%%%%%%%%%%%%%

\usepackage{multirow}		% Zeilen in Tabellen zusammenfassen (http://tug.ctan.org/tex-archive/macros/latex/required/graphics/grfguide.pdf)
\usepackage{hhline}		% Horizontale Linien in Tabellen feiner gestalten

% Unterabbildungen können mit dem subfig-Paket realisiert werden.
% (z.B. Abb. 1a) (http://tug.ctan.org/tex-archive/macros/latex/contrib/subfig/subfig.pdf)
\usepackage{subfig}


% Sehr gute Möglichkeiten um Grafiken direkt in LaTeX zu erzeugen bietet Tikz (http://mirror.ctan.org/graphics/pgf/base/doc/generic/pgf/pgfmanual.pdf und http://www.statistiker-wg.de/pgf/tutorials.htm)
%\usepackage{tikz}		% 
%\usetikzlibrary[shapes.misc,shapes.geometric,arrows,decorations.pathmorphing,backgrounds,fit,positioning]

%Globale Einstellungen für das Listingspaket
\lstset{numbers=left,
	basicstyle=\ttfamily\small, 		% Setzt den Standardstil
	keywordstyle=\bfseries\small, 		% Setzt den Stil für Schlüsselwörter
	identifierstyle=\bfseries, 		% Identifier fett
	tabsize=2,				% Breite der Tabs
	showstringspaces=false,			% Leerzeichen in Strings nicht anzeigen
	commentstyle=\color{red!40}, 		% Stil für Kommentare
	stringstyle=\color{blue}, 		% Stil für Strings (gekennzeichnet mit "String")
	breaklines=true, 			% Zeilen werden umgebrochen
	numbers=left, 				% Zeilennummern links
	numberstyle=\tiny, 			% Stil für die Seitennummern
	frame=single, 				% Rahmen	
	backgroundcolor=\color{black!10}, 	% Hintergrundfarbe
	captionpos=b,				% Position der Listingunterschrift
	float
}


%%%%%%%%%%%%%%%%%%%%%%%%%%%%%%%%%%%%%%%%%%%%%%%%%%%%%%%%%
%%% Ein paar Hilfskommandos, die nützlich sein können %%%
%%%%%%%%%%%%%%%%%%%%%%%%%%%%%%%%%%%%%%%%%%%%%%%%%%%%%%%%%
\newcommand{\code}[1]{\textsf{#1}}	% Code in Text
\newcommand{\pkg}[1]{\textsf{#1}}	% Paketnamen
\newcommand{\class}[1]{\textsf{#1}} 	% Klassennamen
\newcommand{\idx}[1]{_\text{#1}} 	% schreibt in Formeln einen nicht kursiven Index


%%%%%%%%%%%%%%%%%%%%%%%%%%%%%%%%%%%%%%
%%% Einstellungen für das Dokument %%%
%%%%%%%%%%%%%%%%%%%%%%%%%%%%%%%%%%%%%%

% Der Titel wird hier eingetragen
\title{Beispielvorlage für eine Seminarausarbeitung}

% Der/die Verfasser wird hier angegeben
\author{Arthur Dent, Ford Prefect}


%%%%%%%%%%%%%%%%%%%%%%%%%%%%%%%%%%%%%%%%%%%%%
%%% Hier beginnt das eigentliche Dokument %%%
%%%%%%%%%%%%%%%%%%%%%%%%%%%%%%%%%%%%%%%%%%%%%

\begin{document}
	
	
	%%%%%%%%%%%%%%%%%%
	%%% Titelblatt %%%
	%%%%%%%%%%%%%%%%%%
	\begin{titlepage}
		\begin{center}
			\vspace*{1.5cm}
			\begin{Large}
				\textbf{Universit\"at Osnabr\"uck}
			\end{Large}
			
			\noindent\hrulefill
			\\[3.5cm]
			AUSARBEITUNG \\[1cm]
			zum  \\[1cm]
			\textbf{Proseminar Systemwissenschaft} \\[1.5cm]    % Seminartitel eintragen 
			im Wintersemester 2019/20 \\[1.5cm]   % Semester / Datum eintragen
			Thema: \\[0.5cm]
			\textbf{Anwendung partizipativer Methoden in sozialökologischer Forschung} \\[2cm]        % Titel des eigenen Seminarthemas eintragen
		
		\end{center}
		\vfill
		\begin{flushleft}
			Vorgelegt von: 
			\hfill \parbox{60mm}{Julian Unland} \\  % Eigene Daten einsetzen
			\hfill \parbox{60mm}{123456} \\
			\hfill \parbox{60mm}{Albrechtstraße 1} \\
			\hfill \parbox{60mm}{49074 Osnabrück} \\
			\hfill \parbox{60mm}{bsp@uos.de} \\
			\hfill \parbox{60mm}{Martin ben Ahmed} \\  % Eigene Daten einsetzen
			\hfill \parbox{60mm}{970697} \\
			\hfill \parbox{60mm}{Ohnesorgestr. 24} \\
			\hfill \parbox{60mm}{49080 Osnabrück} \\
			\hfill \parbox{60mm}{mbenahmed@uos.de}
		\end{flushleft}
	\end{titlepage}
	%%% end title page
	
	
	%%%%%%%%%%%%%%%%%%%%%%%%%%%%%%%%%%%%%%%%%%%%%%%%%%%%%
	%%% Inhaltsverzeichnis mit römischen Seitenzahlen %%%
	%%%%%%%%%%%%%%%%%%%%%%%%%%%%%%%%%%%%%%%%%%%%%%%%%%%%%
	\newpage
	\pagenumbering{roman}
	\tableofcontents 
	
	
	%%%%%%%%%%%%%%%%%%%%%%%%%%%%%%%%%%%%%%%%%%%%%%%%%%%%%%%%%%%%%%%
	%%% Start des Dokumenteninhalts mit arabischen Seitenzahlen %%%
	%%%%%%%%%%%%%%%%%%%%%%%%%%%%%%%%%%%%%%%%%%%%%%%%%%%%%%%%%%%%%%%
	
	\newpage
	\pagenumbering{arabic}
	
	\maketitle
	
	\begin{abstract}
		Abstract...

	\end{abstract}
	
	\section{Einführung}
		
	\subsection{Zweite Gliederungsebene}
	
	\subsubsection{Dritte Gliederungsebene}
		
	\section{Zweite Section...}
		
	\subsection{Zweite Gliederungsebene}

	\subsubsection{Dritte Gliederungsebene}
	
	\section{Zusammenfassung}
	
	%%%%%%%%%%%%%%%%%%%%%%%%%%%%%%
	%%%% Literaturverzeichnis %%%%
	%%%%%%%%%%%%%%%%%%%%%%%%%%%%%%
	
	\newpage
	
	%\bibliographystyle{./plaindin} % Hier die .bst Datei aus der Vorlage verwenden und in das Verzeichnis mit der Hauptdatei legen.                             
	
	\bibliographystyle{plaindin} % Hier die .bst Datei aus der Vorlage verwenden und in das Verzeichnis mit der Hauptdatei legen.
	% \bibliographystyle{plain}     % andere bibstyles (bst) formatieren 
	%\bibliographystyle{alpha}     % die Literatur anders
	\bibliography{bspbib} % Bibtex Dateiname einsetzen (enthält die Literaturquellen)
	
	
	
	%%%%%%%%%%%%%%%%%%%%%%%%%%%
	%%% Anhang zum Dokument %%%
	%%%%%%%%%%%%%%%%%%%%%%%%%%%
	\appendix	% Ab hier ist Anhang
	\newpage
	
	
	\section{Anhang}
	
	\subsection{Die Zweite Gliederungsebene im Anhang}
	
	
\end{document}